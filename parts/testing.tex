\documentclass[border = 10pt]{standalone}
\usepackage{tikz}
\usetikzlibrary{calc}
\usetikzlibrary{labglass}
\begin{document}

\begin{tikzpicture}

%  \path
%              (0,0) pic (LL) {quadrant III}
%        -- ++ (1,0) pic (LR) {quadrant IV}
%        -- ++ (0,1) pic (UR) {quadrant I}
%        -- cycle
%        -- ++ (0,1) pic (UL) {quadrant II}
%      ;
%  \draw[blue] (LR-ne) -- (LR-sw);
%
% \draw[red] (LL-ne) rectangle (LL-sw);
%
%  \draw (LL-center) -- (UR-center)
%                    -- (UL-center)
%                    -- (LR-center);

	%\draw[green] (0,0) pic (b) {_tapered_open};
	%\draw[red] (0, 0) pic (c) {tapered_stopper};
		%\pic (A) {rbf={90, 45}};
		%\draw[red, fill=red] (A-taper90-center) circle (0.01);
	\def\taperwidth{14}
	\def\taperlength{20}
	\pic (A) {rbf={90, 45}};	
	\pic at (A-taper90-center) (B) {leibig_condenser={150}};
	%\pic[rotate=45-90] at (A-taper45-center) (C) {penny_stopper};
		\node[blue] at (B-taperM-center) (z) {.};
		\node[red] at (B-center) (y) {.};
		\node[green] at (B-taperF-center) (x) {.};
%		\pic[green] at (B-taperF-center) (w) {penny_stopper};
	%\pic[rotate=90-90] at (A-taper90-center) (x) {penny_stopper};
	%\pic[rotate=90-90] at (A-taper90-center) (y) {tapered_stopper};
	
	%\node[green] at (x-center) (c) {.};
	%\draw (At90) -- (At200);
\end{tikzpicture}

\end{document}
