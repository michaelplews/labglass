\def\aeunitcircleradius{1cm}
\def\taperangle{4}
\def\R{0.75}				%radius of flask
\def\neck{3}			%number of necks
\def\pos{90}			%position of first neck in degrees
\def\ang{45}			%separation of necks in degrees
\def\taperwidth{24}	%Width of the joint taper
\def\taperlength{40}	%Length of the joint taper
\usetikzlibrary{calc}
\tikzset{
  quadrant I/.pic={%%
    \draw[line width=1pt] (-\aeunitcircleradius/2,\aeunitcircleradius/2)  arc (90:0:\aeunitcircleradius);
    \path (0,0)                  coordinate (-center)
                -- ++ (0.5,-0.5) coordinate (-se)
                -- ++ (0,1.0)    coordinate (-ne)
                -- ++ (-1,0)     coordinate (-nw)
                -- ++ (0,-1)     coordinate (-sw);
    },
  quadrant II/.pic={%%
    \draw[line width=1pt] (-\aeunitcircleradius/2,-\aeunitcircleradius/2)  arc (180:90:\aeunitcircleradius);
    \path (0,0)                  coordinate (-center)
                -- ++ (0.5,-0.5) coordinate (-se)
                -- ++ (0,1.0)    coordinate (-ne)
                -- ++ (-1,0)     coordinate (-nw)
                -- ++ (0,-1)     coordinate (-sw);
    },
  quadrant III/.pic={%%
    \draw[line width=1pt] (-\aeunitcircleradius/2,\aeunitcircleradius/2)  arc (180:270:\aeunitcircleradius);
    \path (0,0)                  coordinate (-center)
                -- ++ (0.5,-0.5) coordinate (-se)
                -- ++ (0,1.0)    coordinate (-ne)
                -- ++ (-1,0)     coordinate (-nw)
                -- ++ (0,-1)     coordinate (-sw);
    },
  quadrant IV/.pic={%%
    \draw[line width=1pt] (-\aeunitcircleradius/2,-\aeunitcircleradius/2)  arc (-90:0:\aeunitcircleradius);
    \path (0,0)                  coordinate (-center)
                -- ++ (0.5,-0.5) coordinate (-se)
                -- ++ (0,1.0)    coordinate (-ne)
                -- ++ (-1,0)     coordinate (-nw)
                -- ++ (0,-1)     coordinate (-sw);
    },
}
\def\adjtaperlength{\taperlength/100}
\def\adjtaperwidth{\taperwidth/100}
\tikzset{%
	pics/_tapered_open/.style={%%
		code={%%%
	\draw (0,0)
		   ++ (0.5*\adjtaperwidth, 0.5*\adjtaperlength)
		-- ++ ({270-\taperangle}:{\adjtaperlength/cos(\taperangle)})
	       ++ (180:{\adjtaperwidth-(2*\adjtaperwidth*tan(\taperangle))})
		-- ++ ({90+\taperangle}:{\adjtaperlength/cos(\taperangle)});
	\path (0,0)					coordinate (-center)
		-- ++ (0.5*\adjtaperwidth, 0.5*\adjtaperlength) coordinate (-ne)
		-- ++ ({270-\taperangle}:{\adjtaperlength/cos(\taperangle)}) coordinate (se)
		-- ++ (180:{\adjtaperwidth-(2*\adjtaperwidth*tan(\taperangle))}) coordinate (sw)
		-- ++ ({90+\taperangle}:{\adjtaperlength/cos(\taperangle)}) coordinate (nw);
		}%%%
	},%%
}%
\tikzset{%
	pics/tapered_stopper/.style={%%
		code={%%%
	\draw (0, 0)
		   ++ (0.5*\adjtaperwidth, 0.5*\adjtaperlength)
		-- ++ ({270-\taperangle}:{\adjtaperlength/cos(\taperangle)})
		-- ++ (180:{\adjtaperwidth-(2*\adjtaperwidth*tan(\taperangle))})
		-- ++ ({90+\taperangle}:{\adjtaperlength/cos(\taperangle)})
		-- cycle;
	\path (0,0)                  coordinate (-center);
		}%%%
	},%%
}%
\def\pennyanga{100}
\def\pennyrad{0.3958*\adjtaperwidth} %What is the mathematical significance of this? 
\tikzset{%
	pics/penny_stopper/.style={%%
		code={%%%
	\draw (0, 0)
		   ++ (0.5*\adjtaperwidth, 0.5*\adjtaperlength)
		-- ++ ({270-\taperangle}:{\adjtaperlength/cos(\taperangle)})
		-- ++ (180:{\adjtaperwidth-(2*\adjtaperwidth*tan(\taperangle))})
		-- ++ ({90+\taperangle}:{\adjtaperlength/cos(\taperangle)})
		-- cycle;
	\draw (0, 0)
		   ++ (0.5*\adjtaperwidth, 0.5*\adjtaperlength)
		arc (250:250-\pennyanga:\pennyrad)
		arc (-30:{-30+3*(180-\pennyanga)}:\pennyrad)
		arc (30:30-\pennyanga:\pennyrad);
		
	\path (0,0)                  coordinate (-center);
		}%%%
	},%%
}%
\tikzset{%
	pics/rbf/.style={%%
		code={%%%
	\foreach \i in {#1}	
		\draw (\i:{\R+0.5*\adjtaperlength}) pic[rotate={\i-90}] (t\i) {_tapered_open};
	\draw (0, 0) circle (\R);
	%\draw (0,0) -- (t90-ne);
	\path (0,0)										coordinate (-center);
	\foreach \i in {#1}
		\path (\i:{\R+0.5*\adjtaperlength})	coordinate (-taper\i-center);
		}%%%
	}%%
}%
