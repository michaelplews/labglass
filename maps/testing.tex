\documentclass[border = 10pt]{standalone}
\usepackage{tikz}
\usetikzlibrary{calc}
\usetikzlibrary{labglass}
\usetikzlibrary{positioning}
\begin{document}

\begin{tikzpicture}

%  \path
%              (0,0) pic (LL) {quadrant III}
%        -- ++ (1,0) pic (LR) {quadrant IV}
%        -- ++ (0,1) pic (UR) {quadrant I}
%        -- cycle
%        -- ++ (0,1) pic (UL) {quadrant II}
%      ;
%  \draw[blue] (LR-ne) -- (LR-sw);
%
% \draw[red] (LL-ne) rectangle (LL-sw);
%
%  \draw (LL-center) -- (UR-center)
%                    -- (UL-center)
%                    -- (LR-center);

	\def\taperwidth{14}
	\def\taperlength{20}
	\pic (C) {rbf={90, 45}};
	\pic at (C-taper90-center) (A) {leibig_condenser={150}};
	\fill[fill=green!20, opacity=0.5]
		(C-center)
		++ (180:\R)
		arc (180:360:\R)
		-- ++ (180: {2*\R});
	\pic at (A-taperF-center) (B) {pipe_male={50}{12}};
		\coordinate (z) at (B-taperM-center) {};
			\draw[red, fill=red] (z) circle (0.1);
			\node[red, right=1.75cm of z] (z2) {\textbf{\Large-taperM-center}};	
		\coordinate (y) at (B-center) {};
			\draw[blue, fill=blue] (y) circle (0.1);
			\node[blue, right=1.75cm of y] (y2) {\textbf{\Large-center}};	
		\coordinate (x) at (B-end) {};
			\draw[green, fill=green] (x) circle (0.1);
			\node[green, right=1.75cm of x] (x2) {\textbf{\Large-end}};
	\pic[line width=1.5pt] at (B-end) (C) {elbow={335}{12}};	

\end{tikzpicture}

\end{document}
